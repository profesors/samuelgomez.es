%\documentclass[12pt,monochrome]{book}
\documentclass[12pt]{book}
\usepackage{mathtools}
\usepackage[utf8]{inputenc}                         % UNICODE
\usepackage[T1]{fontenc}                            % Permite copiar caractéres unicode desde el PDF al portapapeles
\usepackage{amsmath,amsthm}                         % Matemáticas American Mathematical Society
\usepackage{amssymb}                                % Math symbols
\usepackage{graphicx}                               % Insertar gráficos
\usepackage[ddmmyyyy]{datetime}                     %
\usepackage[spanish, es-nodecimaldot]{babel}        % Corte de las palabras en español
\usepackage{tikz}
\usetikzlibrary{babel}								% Imprescindible para que tikz funcione con babel
\usetikzlibrary{mindmap,trees}
\usepackage{enumitem}
\setenumerate[0]{label=\alph*)}
\usepackage[hidelinks]{hyperref}                    % Enlaces sin apariencia fea
\usepackage{gensymb}                                % Símbolos de grado
\usepackage[pscoord]{eso-pic}                       % Overlay text. The zero point of the coordinate system is the lower left corner of the page (the default).
\usepackage{xcolor}                                 % Texto con diferentes colores
%\selectcolormodel{gray}
\usepackage{siunitx} \sisetup{ output-decimal-marker={,}, quotient-mode=fraction}   % Sistema internacional para unidades con separador decimal
\usepackage{relsize}								% Integrales grandes
\usepackage{qrcode}
\usepackage{sectsty}								% Estilo para las secciones
\usepackage{empheq}									% Ecuaciones en cajas
\usepackage[pagestyles,innermarks]{titlesec}		% Configuración de títulos
\usepackage{float}									% Posición figuras con [H]
%\usepackage{chngcntr}								% Resetea contador de ecuaciones
%\usepackage{xpatch}
%\usepackage{everyshi}								% Contador se incrementa al cambiar de página
\usepackage{imakeidx}
\makeindex

% == Data of the document ==
\date{\today}
\author{Samuel Gómez Fernández}

% == Teoremas ==
\newtheoremstyle{named}{\topsep}{\topsep}{\itshape}{}{\bfseries}{.}{.5em}{\thmname{#1}
	\roman{chapter}\thmnumber{#2} \thmnote{#3}}
\theoremstyle{named}
\theoremstyle{plain}
	\newtheorem{propo}{Proposición}[section]
	\newtheorem{teor}[propo]{Teorema}
	\newtheorem{coro}[propo]{Corolario}
	\newtheorem{lema}[propo]{Lema}
\theoremstyle{definition}
	\newtheorem{defi}{Definición}[chapter]
	\newtheorem{ejem}{Ejemplo}
	\newtheorem{ejer}{Ejercicio}
\theoremstyle{remark}
	\newtheorem*{nota}{Nota}
	\newtheorem*{notac}{Notación}

% Cabecera ===========================================
\newpagestyle{DocumentoGeneral}{
	\headrule
	\sethead{
		\ifnum\value{chapter}>0
			Capítulo \roman{chapter} 
		\else
		\fi
		\hfill \MakeUppercase{\chaptertitle}
		}{}{}
	\footrule\setfoot{}{\usepage{}}{}
	}


% Redefino nombre y número de figura ==================
\renewcommand\thefigure{\roman{chapter}.\arabic{figure}}
%\counterwithin{equation}{section}

% Bibliografía ========================================
\usepackage[backend=biber,style=alphabetic,sorting=nty,citestyle=alphabetic]{biblatex}
\bibliography{bib/matematicas.bib}{}
\setlength\bibitemsep{1.5\itemsep}
\DefaultInheritance{all=true}
%\setlength\bibitemsep{\baselineskip}
\usepackage{authblk}	% Multiple authors

% Entorno problema ========================================
\newcounter{problema}[section]
\newenvironment{problema}[2]
{\refstepcounter{problema} \noindent\textbf{Problema \thesection.\theproblema} \hspace{0.5cm} #2 
	\begin{flushright}\textit{#1}\end{flushright} 
	\paragraph{Solución}}
{\vspace{1cm}}

% Fórmulas enfatizadas ======================================
\definecolor{formulaFondo}{HTML}{e0e0ff}
\newcommand*\formulaBox[1]{\colorbox{formulaFondo}{\hspace{1em}#1\hspace{1em}}}

% Entorno lista de propiedades =================================
\newlist{propiedades}{enumerate}{2}
\setlist[propiedades]{label=Propiedad \arabic*.,itemindent=*}

% Formato de capítulos =========================================
\definecolor{tituloColor}{HTML}{006aff}
%\newcommand{\chapterbreak}{\clearpage}

\newcommand{\chapterFormato}[1]{
	\huge\textcolor{tituloColor}{#1}\\
	\vspace{1cm}
	\color{tituloColor}\titlerule[1pt]
}
\titleformat{\chapter}[display]{\filright\normalfont\bfseries}
	{\textcolor{tituloColor}{	% El número de la sección
		Capítulo \roman{chapter}}\;\;
		\color{tituloColor}\titlerule[1pt]
		%\setcounter{pageInChapter}{1}
	}	
	{0em}	% Distancia entre número y texto de título de sección
	{\chapterFormato}	% El texto de la sección


% Formato de secciones =========================================
\newcommand{\sectionFormato}[1]{
	\large\textcolor{tituloColor}{#1}
}
\titleformat{\section}[hang]{\normalfont\bfseries}
	{\textcolor{tituloColor}{	% El número de la sección
		\Large\thesection.}
	}	
	{0em}	% Distancia entre número y texto de título de sección
	{\sectionFormato}	% El texto de la sección
\renewcommand{\thesection}{\arabic{section}}

% Formato de subsecciones =========================================
\newcommand{\subsectionFormato}[1]{
	\large\textcolor{black}{#1}
}
\titleformat{\subsection}[hang]{\normalfont\bfseries}
	{\textcolor{black}{	% El número de la sección
		\large\thesubsection.}
	}	
	{0em}	% Distancia entre número y texto de título de sección
	{\subsectionFormato}	% El texto de la sección
% QR CODE ===========================================
\qrset{height=2cm,padding}


\usepackage[a4paper,top=3cm,right=3cm,bottom=3cm,left=3cm]{geometry}    % Cambiar márgenes del documento
\setlist[0]{leftmargin=1cm,labelindent=\parindent, itemindent=0cm}		% Solo el primer nivel
\setlength{\parskip}{0.4cm}	% Espacio entre párrafos

\title{Tratado de álgebra}

%\renewcommand\vec{\mathbf}
%index{término para el índice}
\renewcommand\vec{\overline}
\begin{document}

\maketitle
\tableofcontents
\newpage

%==========================================
\chapter{Espacios vectoriales}
Sea un conjunto de elementos denominados vectores y otro conjunto de elementos denominados
escalares donde los primeros tienen una operación interna y otra externa tal que cumplen

\paragraph{Ley interna +} configura la estructura $(V,+)$ como grupo abeliano y por lo tanto
cumple las propiedades $AesC$
\begin{propiedades}
\item Asociativa $\forall\vec{x},\vec{y},\vec{z}\in
	V(k)\Rightarrow(\vec{x}+\vec{y})+\vec{z}=\vec{x}+(\vec{y}+\vec{z})$
\item Elemento neutro $\forall\vec{x}\in V(k), \exists\vec{0}\in V(k) /  \Rightarrow
	\vec{x}+\vec{0}=\vec{x}$
\item Simetrizable $\forall \vec{x}\in V(k), \exists(-\vec{x})\in V(k) \Rightarrow
	\vec{x}+(-\vec{x})=\vec{0}$
\item Conmutativa $\forall\vec{x},\vec{y}\in V(k)\Rightarrow\vec{x}+\vec{y}=\vec{y}+\vec{x}$
\end{propiedades}

\paragraph{Ley externa $\cdot$} denominada producto por un escalar que asocia a cada vector un
escalar de los conjuntos antes mencionados de la siguiente manera $\lambda\vec{x}=\vec{x}\lambda$
y cumple las propiedades. $AeD^+_vD^+_c$

\begin{propiedades}
\item Asociativa mixta $\alpha(\beta\vec{x})=(\alpha\beta)\vec{x}$
\item Elemento neutro Si $1$ es el elemento neutro de la ley externa $1\cdot\vec{x}=\vec{x}$
\item Distributiva respecto de la suma de vectores
	$\lambda(\vec{x}+\vec{y})=\lambda\vec{x}+\lambda\vec{y}$
\item Distributiva respecto de la suma de escalares
	$(\alpha+\beta)\vec{x}=\alpha\vec{x}+\beta+\vec{x}$
\end{propiedades}

\section{Ejemplos de espacios vectoriales}
\begin{enumerate}
	\item Vectores libres tridimensionales sobre el cuerpo $\mathbb{R}$, con las operaciones:
		\paragraph{Suma} Aplicando la regla del paralelogramo $\vec{x}+\vec{y}$.
		\paragraph{Producto por un número real} El producto de un número real $\lambda$ por un
		vector $\vec{x}$ es otro vector $\lambda\vec{x}$, cuyo módulo se obtiene multiplicando por
		$|\lambda|$ el módulo de $\vec{x}$, su dirección es la de $\vec{x}$, y su sentido coincide
		con el de $\vec{x}$ cuando $\lambda$ es positivo, y el contrario cuando es negativo.
	\item El conjunto $\mathbb{R}(\mathbb{R})$
		\paragraph{Suma} Estos vectores unidimensionales son fáciles de operar
		$\mathbb{R}\times\mathbb{R}\rightarrow\mathbb{R}$ puesto que solo tienen un componente. Por
		ejemplo $\vec{3}+\vec{5}=\vec{8}$
	\item Producto por un escalar $\mathbb{R}\times\mathbb{R}\rightarrow\mathbb{R}$. Por ejemplo
		$3\cdot\vec{5}=\vec{15}$.
\end{enumerate}

\section{Propiedades que se deducen de los axiomas}
\begin{enumerate}
	\item $\forall \lambda\in\mathbb{K}, \lambda\vec{0}=\vec{0}$ \\
		Demostración
		$\lambda\vec{x}=\lambda\cdot(\vec{x}+\vec{0})=\lambda\vec{x}+\lambda\vec{0} \Rightarrow
		\lambda\vec{0}=\vec{0}$
	\item $\forall \vec{x}\in V(\mathbb{K}), 0\vec{x}=\vec{0}$ \\
		Demostración
		$\vec{x}=\vec{x}(1+0)=\vec{x}+0\vec{x}\Rightarrow 0\vec{x}=\vec{0}$
	\item Si $\lambda\vec{x}=\vec{0}\Rightarrow\lambda=0 \lor \vec{x}=\vec{0}$ \\
		Para su demostración hay dos posibilidades:
		
		Primera posibilidad $\lambda\neq 0 \Rightarrow \exists\lambda^{-1}$ porque $\lambda\in\mathbb{K}$ y
		podemos escribor $\vec{x}=\lambda^{-1}(\lambda\vec{x})$ sustituyendo con el antecedente 
		$\lambda^{-1}\vec{0} \Rightarrow (\lambda^{-1}\lambda)\vec{x}=\vec{0}$ y como lo que está
		entre paréntesis es $1$, nos queda que $1\vec{x}=\vec{0}$, es decir $\vec{x}=\vec{0}$.\\
		
		Segunda posibilidad $\lambda=0$ ya demostado en $0\vec{x}=\vec{0}$.

	% En Palacios hay tres propiedades más en la página 293

\end{enumerate}
% ================================= SUBESPACIOS VECTORIALES ================================
\chapter{Subespacios vectoriales}
\section{Propiedades útiles de subespacios vectoriales}
\begin{propiedades}
	\item $\lambda F=F$ Siendo $F$ un subespacio vectorial y $\lambda$ un escalar perteneciente a un
		$\mathbb{K}$. La demostración queda expresada en el ejercicio \thesection.\ref{20200413a}.
	\item $\mathbb{K}(\lambda F)=\mathbb{K}(F)$
		Queda demostrado en el ejercicio \thesection.\ref{20200414a}
	\item $\mathbb{K}\vec{u} \subseteq F \iff \vec{u} \in F$
		Demostración en \ref{20200414b}
\end{propiedades}

\section{Dimensiones}
Es el número de vectores de la base. También se corresponde con el el número de parámetros 
del sistema como por ejemplo en ${(x,y,z)=\lambda(2,-1,0)+\mu(3,0,-1)}$ donde la dimensión sería 2.

\section{Suma de subespacios vectoriales}

Sean $A$ y $B$ dos subconjuntos no vacíos de $E$ se define $A+B$ como el conjunto formado por la
suma de todos los vectores pertenecientes a $A$ con todos los pertenecientes a $B$, es decir

$$A+B = \{\vec{x}+\vec{y}/\vec{x}\in A, \vec{y}\in B\}$$

\paragraph{Subespacios lineales independientes} si todo vector $\vec{v}$ se puede escribir de manera
única como $\vec{v}=\vec{v}_1+\vec{v}_2+ ... +\vec{v}_n$ con $\vec{v}_1\in F_1, \vec{v}_2\in F_2,
..., \vec{v}_n\in F_n$. O visto del modo de Kostrikin\footnote{\cite[p60]{kostrikin}},
si $\vec{v}_1\cap\vec{v}_2\cap...\vec{v}_n=\emptyset$.

\paragraph{Suma directa} Si $n$ subespacios vectoriales son independientes, su suma se llama suma
directa y la denotamos como $\oplus$, como por ejemplo $F_1 \oplus F_2 \oplus F_3$.

\paragraph{Espacios vectoriales suplementarios} Si dos subespacios vectoriales son independientes y
su suma directa es $E$, es decir $F\oplus G = E$

% ====================================== HOMOMORFISMOS ======================================
\chapter{Homomorfismo entre espacios vectoriales}
Sean $V(k)$ y $W(k)$ dos espacios vectoriales definidos sobre el mismo cuerpo $k$ de operadores, y
sea $f$ una aplicación de $V$ en $W$. Diremos que $f$ es un \textbf{homomorfismo o aplicación
lineal}, si se cumplen las dos condiciones:

$f(\vec{x}+\vec{y})=f(\vec{x})+f(\vec{y})$

$f(\lambda\vec{x}) = \lambda f(\vec{x})$

\section{Imagen de un homomorfismo}
Conjunto formado por 
$$Im(f)=\{\vec{y}\in W/\exists \vec{x}\in V, \vec{y}=f(\vec{x})\}$$

\section{Núcleo de un homomorfismo}
Dado un Homomorfismo $f$ de $V(k)$ en $W(k)$, se denomina núcleo al conjunto de 
todos los vectores de $V$ cuya imagen es el vector nulo de $W$.

$$N(f)=\{\vec{x}\in V/f(\vec{x})=\vec{0}_w\}$$

% ======================= PROBLEMAS DE ESPACIOS VECTORIALES==================================
\chapter{Problemas}

\section{Problemas de espacios vectoriales}

% ========================== 90 p 140
\begin{problema}{\cite[90p140]{prieto2000}}{
	Sean $\vec{u}$ y $\vec{v}$ dos vectores de un espacio vectorial $E$ sobre un cuerpo
	$\mathbb{K}$, y sea $\mathbb{K}^*=\mathbb{K}-\{0\}$. Desmostrar la equivalencia
	$$(\mathbb{K}\vec{u}=\mathbb{K}\vec{v}) \iff
	(\exists\lambda\in\mathbb{K}^*,\vec{u}=\lambda\vec{v})$$
	}
	Si $\vec{u}$ y $\vec{v}$ son nulos es trivial. Para el resto de casos:

	Si $\mathbb{K}\vec{u}=\mathbb{K}\vec{v}$ es $\forall a,b \in \mathbb{K}, a\vec{u}=b\vec{v}$ es
	decir $\vec{u}=\frac{b}{a}\vec{v}=\lambda\vec{v}$.

	Otra forma de ver esto mismo es $\mathbb{K}\vec{u}=\mathbb{K}\vec{v}$ implica que $\mathbb{K}\vec{u}\subseteq
	\mathbb{K}\vec{v}$ y por lo tanto $\vec{u}\in\mathbb{K}\vec{u}\subseteq\mathbb{K}\vec{v}$ para
	dar lugar a $\vec{u}\in\mathbb{K}\vec{v}$ lo cual implica que hay algún valor $\lambda$ tal que
	$\vec{u}=\lambda\vec{v}$.

	Por otra parte, si $\exists\lambda\in\mathbb{K}^*, \vec{u}=\lambda\vec{v}$ significa que 
	para la segunda parte, si
	$\vec{u}=\lambda\vec{v}\Rightarrow(\mathbb{K}\vec{u}=\mathbb{K}\vec{v})$ 
	podemos sustituir $\vec{u}=\lambda\vec{v}$ y obtenemos
	$\mathbb{K}\vec{u}=\mathbb{K}\lambda\vec{u}=\mathbb{K}(\lambda\vec{u})=\mathbb{K}\vec{v}\;\blacksquare$
\end{problema}

% ================ 3p334
\begin{problema}{\cite[3p334]{palacios}}{
	En $\mathbb{R}^4(\mathbb{R})$ se considera el sistema de vectores
	$$s=\{(17,8,9,12),(0,0,1,5),(0,2,1,1)\}$$
	Averiguar si es un sistema libre
	}
	$$Rango\begin{pmatrix}
		17 & 8 & 9 & 12 \\
		0 & 0 & 1 & 5 \\
		0 & 2&1&1
	\end{pmatrix}=3 \ne 0$$
	Luego las tres son linealmente independientes y sí es un sistema libre
\end{problema}

%================= PROBLEMAS DE SUBESPACIOS VECTORIALES =========================================
\section{Problemas de subespacios vectoriales}
% ==== 87p139
\begin{problema}{\cite[87p139]{prieto2000}}{\label{20200413a}
	Considerése un subespacio vectorial $F$ de un espacio vectorial $E$, y sea $\lambda$ un escalar.
	Designemos por $\lambda F$ el siguiente conjunto:
	$$\lambda F=\{\lambda\vec{x}/\vec{x}\in F\}$$
	Desmostrar que si $\lambda$ es no nulo, entonces $\lambda F=F$
	}
	Este problema es muy interesante porque permite emplear una herramienta muy potente al igualar
	los conjuntos $\lambda F = F$ para posteriores ejercicios.
	
	La igualdad implica $\lambda F\subseteq F \land F\subseteq \lambda F$. Atacamos el ejercicio por
	partes.
	\paragraph{Parte 1} $\lambda F\subseteq F \Rightarrow \lambda F = \{\lambda \vec{x}/\vec{x}\in
	F\} \subseteq F$ Como $F$ es un subespacio vectorial cumple siempre $\forall \lambda \in
	\mathbb{K}$ por lo tanto, el conjunto $\lambda F$ está incluido en $F$.

	\paragraph{Parte 2} $\forall \vec{x}\in F, \vec{x}=\lambda(\lambda^{-1}\vec{x})=\lambda\vec{y}$
	donde el vector $\vec{y}$ pertenece a $F$, es decir $\vec{y}\in F$  por lo que $\lambda\vec{y}\in \lambda
	F$ y como podemos sustituir $\vec{x}=\lambda\vec{y}\in \lambda F$ es decir $F \subseteq \lambda
	F$ porque $\vec{x}$ representa todos los vectores de $F$ como indicamos al principio del
	párrafo. $\blacksquare$

	\paragraph{Conclusión} $\lambda F = F$ está formado por los representantes de clase y $\lambda$
	solo está dando como resultado vectores simplificados o aumentados.
\end{problema}

% ========================== 92 p 141
\begin{problema}{\cite[92p141]{prieto2000}}{\label{20200414b}
	Considérese un subespacio vectorial $F$ de un espacio vectorial $E$ sobre un cuerpo
	$\mathbb{K}$, y sea $\vec{u}$ un vector de $E$. Demostrar se verifica:
	\begin{enumerate}
		\item El subespacio vectorial $\mathbb{K}\vec{u}$ está contenido en $F$ si y solo si
			$\vec{u}\in F$ en símbolos: $\mathbb{K}\vec{u}\subseteq F \iff \vec{u}\in F$
		\item Si $\vec{u}\not\in F$, entonces $\mathbb{K}\vec{u} \cap F = \{\vec{0}\}$
	\end{enumerate}
	}
	\paragraph{a.} Si $\mathbb{K}\vec{u}\subseteq F \Rightarrow \vec{u}\in F$ podemos deducir que
	$1\cdot\vec{u}=\vec{u}\in F$ que es lo que buscábamos. Por otra parte, para $\vec{u}\in F$
	podemos ver que como $F$ es subespacio, por su definición se cumple.

	\paragraph{b.} Si $\vec{u}\not\in F \Rightarrow \mathbb{K}\vec{u}\cap F=\{\vec{0}\}$ supongamos
	$\vec{u}\not\in F$. Entonces, quiera que sea el escalar $\lambda$ no nulo, se tiene que
	$\lambda\vec{u}\not\in F$, pues en caso contrario, al ser $F$ subespacio vevtorial de $E$, se
	deduciría la pertenencia a $F$ del vector $\vec{u}=\lambda^{-1}(\lambda\vec{u})$. En consecuencia, el
	único vector que pertenece simultaneamente a $\mathbb{K}\vec{u}$ y a $F$ es el $\vec{0}$.
\end{problema}
% =========================== 88 p 139
\begin{problema}{\cite[88p139]{prieto2000}}{\label{20200414a}
	Si $\vec{v}$ es un vector de un espacio vectorial $E$ sobre un cuerpo $\mathbb{K}$, y $\lambda$ es
	un escalar, demostrar que le conjunto $\lambda\mathbb{K}\vec{v}$ es igual al subespacio
	vectorial $\mathbb{K}(\lambda\vec{v})$. Si $\lambda$ es no nulo, deducir la igualdad:
	$\mathbb{K}(\lambda\vec{v})=\mathbb{K}\vec{v}$.
	}
	\paragraph{Parte 1} Tenemos 
	$\lambda\mathbb{K}\vec{v}=\{\lambda\vec{x}/\vec{x}\in\mathbb{K}\vec{v}\}
		=\{\lambda(\mu\vec{v})/\mu\in\mathbb{K}\}=\{\mu(\lambda\vec{v}/\mu\in\mathbb{K}\}
		=\mathbb{K}(\lambda\vec{v}) \; \blacksquare$
	
	\paragraph{Parte 2} Si $\lambda\neq0$ podemos emplear lo demostrado en el ejercicio
	\thesection.\ref{2020041304a} para escribir
	$\mathbb{K}(\lambda\vec{v})=\lambda\mathbb{K}\vec{v}=\mathbb{K}\vec{v}$.

	Hemos aprovechado que $\mathbb{K}$ cumple la propiedad conmutativa de la segunda ley.
\end{problema}
\begin{problema}{\cite[4p335]{palacios}}{
	En el espacio vectorial $\mathbb{R}^3(\mathbb{R})$ se considera el subconjunto
	$$S=\{(x,y,z)\in \mathbb{R}^3 / x+2y+3z=0\}$$
	\begin{enumerate}
		\item Demostrar que $S$ es subespacio de $\mathbb{R}^3$
		\item Determinar una base para el subespacio
		\item Expresar el vector $\vec{a}=(-21,3,5)$ con dicha base
	\end{enumerate}
	}
	\paragraph{a. Demostrar que $S$ es subespacio} Cumple las tres condiciones
	\paragraph{b. Determinar una base} Tenemos una ecuación implícita para un total de 
	tres componentes, luego necesito
	dos parámetro, o lo que es lo mismo, una base con dos vectores.

	Busco dos vectores linealmente independientes que cumplan le ecuación implícita
	$\vec{e_1}=(2,-1,0)$ y $\vec{e_2}=(3,0,-1)$

	\paragraph{c. Expresar el vector $\vec{a}=(-21,3,5)$ con dicha base}

	$$(-21,3,5)=a(2,-1,0)+b(3,0,-1)$$

	$$\begin{cases}
		-21 &=2a+3b \\
		\phantom{-2}  3 &=-a \\
		\phantom{-2}  5 &=-b
	\end{cases}$$

	Obtenemos los valores $a=-3$ y $b=-5$ y por lo tanto $\vec{a}=-3\vec{e}_1-5\vec{e}_2$
\end{problema}

% =======================================================
\begin{problema}{\cite[5p336]{palacios}}{
	Determinar una base del subespacio vectorial $\mathbb{R}^3(\mathbb{R})$ cuyos elementos son
	la solución de la ecuación $2x-y+z=0$
	}
	\begin{center}
		Dim espacio vectorial\footnote[1]{Número de componentes} 3 = 1 ec. implícita + 2 Dim
		subespacio
		parámetros\footnote[2]{Número de vectores de la base o número de parámetros}
	\end{center}
	Dos vectores que cumplen la ecuación implícita y que son linealmente independientes 
	para la base son $B=\{(0,1,1),(1,0,-2)\}$
\end{problema}

% ===============================================
\begin{problema}{\cite[6p338]{palacios}}{
	En el espacio vectorial $\mathbb{R}^4(\mathbb{R})$ se tiene el sistema de vectores
	${S=\{(1,0,0,0),(1,1,0,0),(0,2,1,0)\}}$ y el subespacio
	${W=\{(x_1,x_2,x_3,x_4)\in\mathbb{R}^4/x_1+x_2=0\}}$
	\begin{enumerate}
		\item Hallar una base para $W$
		\item Ecuaciones paramétricas de $L(S)$
		\item Obtener una base de $\mathbb{R}^4(\mathbb{R})$ completando la base
			${\{(1,0,0,0),(1,1,0,0),(0,2,1,0)\}}$
		\item Hallar una base para el subespacio $W\cap L(S)$
	\end{enumerate}
	}
	\paragraph{a. Hallar una base para $W$} Un subespacio vectorial está sujeto a las restriccione
	del espacio vectorial al cual pertenece. Los tres vectores de ·S· son linealmente independientes 
	y cumplan la ecuación implícita de $W$, por lo tanto son la respuesta.
	\begin{center}
		Dim espacio vectorial 4 = 1 ec. implícita + 3 Dim subespacio
	\end{center}

	\paragraph{b. Ecuaciones paramétricas} $(x,y,z,t)=\lambda(1,0,0,0)+\beta(1,1,0,0)+\gamma(0,2,1,0)$
	Paramétricamente queda
	$$\begin{cases}
		x &= \lambda+\beta \\
		y &= \beta+2\gamma \\
		z &= \gamma \\
		t &= 0
	\end{cases}$$
	Solo necesito una única ecuación implícita puesto que tengo ya los tres parámetros que implica que la
	base tiene tres vectores. Así que me quedo con $t=0$

	\paragraph{c. Obtener una base} Para completar en $\mathbb{R}^4(\mathbb{R})$ la base
	${B=\{(1,0,0,0),(1,1,0,0),(0,2,1,0)\}}$ solo necesito un vector que será
	$(0,0,0,1)$

	\paragraph{d.} Una base para el subespacio $W\cap L(S)$ se puede obtener de la interección de las ecuaciones
	implícitas de $W$ y $L(S)$.
	$$\begin{cases}
		x_1+x_2 &= 0 \\
		x_4 &= 0
	\end{cases}$$
	\begin{center}
		Dim espacio vectorial 4 = 2 ec. implícitas + 3 Dim subespacio
	\end{center}
	Necesito dos vectores linealmente independientes que cumplan las ecuaciones implícitas
	$$B(W\cap L(S))=\{(1,-1,0,0),(0,0,1,0)\}$$
\end{problema}

% ===================================
\begin{problema}{\cite[8p342]{palacios}}{
	En $\mathbb{R}^5(\mathbb{R})$ se define el subespacio $W$ por sus ecuaciones implícitas
	$$\begin{cases}
		5x_1+x_3-3x_4-x_5 &= 0 \\
		2x_1+2x_2-3x_4 &= 0
	\end{cases}$$
	Determinar su dimensión, unas ecuaciones paramétricas y una base.
	}
	\paragraph{Dimensión} El subespacio es de dimensión 3
	\begin{center}
		Dim espacio vectorial 5 = 2 ec. implícitas + 3 Dim subespacio
	\end{center}

	\paragraph{Ecuaciones paramétricas} Necesitaremos tres parámetros porque el subespacio es de
	dimensión 3. Elijo libremente asociar tres incógnitas a tres parámetros, así $x_3=a$,
	$x_4=b$ y $x_5=c$ quedando
	$$\begin{cases}
		5x_1 &= -x_3+3x_4+x_5 \\
		2x_1+2x_2 &= 3x_4
	\end{cases}
	\;\;\Rightarrow\;\;
	\begin{cases}
		x_1 &= \frac{1}{5}(-a+3b+c) \\
		x_2 &= \frac{1}{10}(2a+9b-2c)
	\end{cases}$$

	\paragraph{Base} Como tengo tres parámetros debo encontrar tres vectores. Voy a activarlos
	asignando valores $0$ y $1$ secuencialmente primero para $a$, luego para $b$ y finalmente para
	$c$. 
	\begin{center}
		%\setlength\extrarowheight{5pt}
		\begin{tabular}{ccc|c}
			\hline
			\textbf{a} & \textbf{b} & \textbf{c} & \textbf{vector} \\
			\hline
			1 & 0 & 0 & $\left(-\frac{1}{5},\frac{1}{5}, 1, 0, 0\right)$ \\[5pt]
			0 & 1 & 0 & $\left(\frac{3}{5},\frac{9}{10}, 0, 1, 0\right)$ \\[5pt]
			0 & 0 & 1 & $\left(\frac{1}{5},-\frac{1}{5}, 0, 0, 1\right)$ \\[5pt]
			\hline
		\end{tabular}
	\end{center}
\end{problema}

% =========================================================
\begin{problema}{\cite[8p345]{palacios}}{
	En un espacio vectorial $V^3(\mathbb{R})$ se tienen dos bases ligadas por la relación matricial
	$$\begin{pmatrix}
		\vec{e'}_1 \\
		\vec{e'}_2 \\
		\vec{e'}_3 \\
	\end{pmatrix}
	=
	\begin{pmatrix}
		1 & 3 & 3 \\
		1 & 4 & 3 \\
		1 & 3 & 4
	\end{pmatrix}
	\begin{pmatrix}
		\vec{e}_1 \\
		\vec{e}_2 \\
		\vec{e}_3 \\
	\end{pmatrix}
	$$
	\begin{enumerate}
		\item Expresar las coordenadas del vector genérico $\vec{x}$ en la nueva base
			$$\vec{x}=x_1\vec{e}_1+x_2\vec{e}_2+x_3\vec{e}_3$$
		\item Cambiar de base el vector $\vec{a}=3\vec{e'}_1-2\vec{e'}_2+\vec{e'}_3$
	\end{enumerate}
	}
	\paragraph{a. Expresar coordenadas en la nueva base} Operando puedo entender mejor 
	la información que me están ofreciendo. Me quedo con las
	igualdades del primer y último miembro de la expresión
	$$\begin{pmatrix}
		\vec{e'}_1 \\
		\vec{e'}_2 \\
		\vec{e'}_3 \\
	\end{pmatrix}
	=
	\begin{pmatrix}
		1 & 3 & 3 \\
		1 & 4 & 3 \\
		1 & 3 & 4
	\end{pmatrix}
	\begin{pmatrix}
		\vec{e}_1 \\
		\vec{e}_2 \\
		\vec{e}_3 \\
	\end{pmatrix}
	=
	\begin{pmatrix}
		\vec{e}_1+3\vec{e}_2+3\vec{e}_3 \\
		\vec{e}_1+4\vec{e}_2+3\vec{e}_3 \\
		\vec{e}_1+3\vec{e}_2+4\vec{e}_3
	\end{pmatrix}
	$$
	En otras palabras, puedo decir que
	$$
	\begin{cases}
		\vec{e'}_1=(1,3,3) \\
		\vec{e'}_2=(1,4,3) \\
		\vec{e'}_3=(1,3,4)
	\end{cases}
	$$
	Por otro lado, la inversa de la matriz de cambio es
	$$A^{-1}=
	\begin{pmatrix}
		\phantom{-}7 & -3 & -3 \\
		-1 & \phantom{-}1 & \phantom{-}0 \\
		-1 & \phantom{-}0 & \phantom{-}1
	\end{pmatrix}
	$$
	Y el cambio para el vector $(a,b,c)$ para obtener un nuevo vector $(x,y,z)$ con el cambio de
	base se realiza así
	$$
	(x,y,z)=(a,b,c)\cdot A^{-1}=
	(a,b,c) \cdot
	\begin{pmatrix}
		\phantom{-}7 & -3 & -3 \\
		-1 & \phantom{-}1 & \phantom{-}0 \\
		-1 & \phantom{-}0 & \phantom{-}1
	\end{pmatrix}
	$$
	$$(x,y,z)=(7a-b-c, -3a+b, -3a+c)$$
	O en paramétricas
	$$
	\begin{cases}
		x &= 7a-b-c\\
		y &= -3a+b\\
		z &= -3a+c
	\end{cases}
	$$
	Que son las coordenadas genéricas del vector $\vec{x}=(a,b,c)$ que me pedían

	\paragraph{b. Cambiar de base el vector} Cambiaré de base 
	a $\vec{a}=3\vec{e'}_1-2\vec{e'}_2+\vec{e'}_3$ pero cuidado que
	en este enunciado no me piden que emplee la fórmula del cambio obtenido en el anterior apartado. 
	Resolveré sustituyendo $\vec{e'}_i$
	$$\vec{a}=3\vec{e'}_1-2\vec{e'}_2+\vec{e'}_3=3(1,3,3)-2(1,4,3)+(1,3,4)=(2,4,7)$$
	Este es el vector cambiado de base.

\end{problema}

% ======= PROBLEMAS HOMOMORFISMOS ============================================
\section{Problemas de homomorfismos}
\begin{problema}{\cite[10p346]{palacios}}{
	Se define la aplicación $f:\mathbb{R}^3\rightarrow\mathbb{R}^3$ dada por
	$f(x_1,x_2,x_3)=(x_1,0,x_1+x_2)$ referida a la base canónica.
	Se pide:
	\begin{enumerate}
	\item Demostrar que $f$ es un homomorfismo
	\item Matriz del homomorfismo
	\item Ecuaciones y dimensiones de $N(f)$
	\item Ecuaciones y dimensiones de $Im(f)$
	\end{enumerate}
	}
	\paragraph{a. Demostrar que es un homomorfismo} Debe cumplir las dos propiedade

	\paragraph{I. Imagen de la suma} Debe cumplir $f(x+y)=f(x)+f(y)$ 
	\begin{align*}
		f[x_1,x_2,x_3)+(y_1,y_2,y_3)] &=f(x_1+y_1,x_2+y_2,x_3+y_3)\\
		&=(x_1+y_1,0,x_1+y_1+x_2+y_2)\\
		&=(x_1,0,x_1+x_2)+(y_1,0,y_1+y_2)\\
		&=f(x_1,x_2,x_3)+f(y_1,y_2,y_3)
	\end{align*}

	\paragraph{II. Imagen del producto por un escalar} Debe cumplir $f(\lambda x)=\lambda f(x)$ 
	\begin{align*}
		f[\lambda(x_1,x_2,x_3)] &= f(\lambda x_1, \lambda x_2, \lambda x_3) \\
		&= (\lambda x_1, 0, \lambda x_1 + \lambda x_2) \\
		&= \lambda(x_1, 0, x_1+x_2) \\
		&= \lambda f(x_1, x_2, x_3)
	\end{align*}

	Luego $f$ es un homomorfismo.

	\paragraph{b. Matriz del homomorfismo} Es la que permite transformar un vector del conjunto
	origen en uno del conjunto imagen. En este caso, la aplicación indica que el vector
	$(x_1,x_2,x_3)
	\rightarrow(x_1,0,x_1+x_2)$. Esto se consigue con la siguiente igualdad

	$$(x_1,x_2,x_3)
	\begin{pmatrix}
		1 & 0 & 1 \\
		0 & 0 & 0 \\
		0 & 0 & 0
	\end{pmatrix}
	=(x_1, 0, x_1+x_2)$$
	
	Luego la matriz del homomorfismo o de transformación es

	$$T=\begin{pmatrix}
		1 & 0 & 1 \\
		0 & 0 & 0 \\
		0 & 0 & 0
	\end{pmatrix}$$

	\paragraph{c. Ecuaciones y dimensiones del núcleo del homomorfismo} También indicado con $N(f)$,
	es la matriz que nos da el vector nulo

	$$(x_1,x_2,x_3)\cdot
	\begin{pmatrix}
		1 & 0 & 1 \\
		0 & 0 & 1 \\
		0 & 0 & 0
	\end{pmatrix}
	=(0,0,0)
	$$

	Se consigue desarrollando el producto de matrices e igualando

	$$(x_1,0,x_1+x_2)=(0,0,0)$$
	Haciendo corresponder componente a componente obtengo las ecuaciones implícitas del núcleo
	$$
	\begin{cases}
		x_1 &=0 \\
		x_1+x_2 &= 0
	\end{cases}
	\quad\Rightarrow\quad
	x_1=x_2=0
	$$
	
	Tengo dos ecuaciones implícitas en un espacio vectorial de dimensión 3. Este valor 3 es la suma
	de las dimensiones del núcleo y de la imagen.\\
	\begin{align*}
		dim(\mathbb{R}^3) &= dim[N(f)]+dim[Im(f)] \quad \scriptstyle{\text{sustituyendo}}\\
		3 &= dim[N(f)] + 2
	\end{align*}
	Donde el primer miembro es la dimensión del conjunto origen y $dim[Im(f)]=Rango(T)=2$. Por lo tanto la dimensión 
	del núcleo del homomorfismo es 1 y el conjunto buscado es
	$N(f)=\{(0,0,\lambda) / \lambda \in \mathbb{R}\}$ y su base está formada por un único vector
	$B_N=\{(0,0,1)\}$.

	\paragraph{d. Ecuación y dimensiones de $Im(f)$} Podemos ver que la expresión
	$f(x_1,x_2,x_3)=(x_1,0,x_1+x_2)$ implica que
	$$
	\begin{cases}
		y_1 = x_1		& \text{paramétrica} \\
		y_2 = 0 		& \text{implícita} \\
		y_3 = x_1+x_2 	& \text{paramétrica}
	\end{cases}
	$$

	Estas son las ecuaciones paramétricas de $Im(f)$, siendo $y_2=0$ su única ecuación implícita.
	En cuanto a su dimensión, tal y como hemos visto en el anterior, tenemos que 
	$dim[Im(f)]=Rango(T)=2$.

	$$Im(f)=\{(\alpha, 0, \beta) / \alpha,\beta \in \mathbb{R}\}$$

	Y su base tendría dos vectores que son dos vectores libres de $T$ tal como
	$B_I=\{(1,0,1),(0,0,1)\}$ que permiten obtener todos los resultados posibles del vector 
	$(y_1,y_2,y_3)$ mediante combinación lineal.
\end{problema}

% =======================================
\begin{problema}{\cite[11p347]{palacios}}{
	En $V^3(\mathbb{R})$ se tiene la base $B_1=\{\vec{e}_1,\vec{e}_2,\vec{e}_3\}$ y en
	$V^2(\mathbb{R})$ la base $B_2=\{\vec{u}_1,\vec{u}_2\}$. Se define la aplicación lineal:
	$$
	\begin{cases}
		f(\vec{e}_1) = \vec{u}_1 + \vec{u}_2 \\
		f(\vec{e}_2) = \vec{u}_1 - \vec{u}_2 \\
		f(\vec{e}_3) = 2\vec{u}_1
	\end{cases}
	$$
	Se pide
	\begin{enumerate}
	\item Expresión matricial del homomorfismo
	\item Ecuación y dimensiones de $N(f)$ y de $Im(f)$
	\item Estudiar de que tipo es la aplicación lineal dada
	\end{enumerate}
	}

	\paragraph{a. Expresión matricial del homomorfismo} Es la matriz de transformación. 
	Lo obtenemos a partir del vector del conjunto origen
	$(x_1,x_2,x_3)=x_1\vec{e}_1+x_2\vec{e}_2+x_3\vec{e}_3$ y su transformación:

	\begin{align*}
		f(x_1,x_2,x_3) 	&= f(x_1\vec{e}_1+x_2\vec{e}_2+x_3\vec{e}_3) \\
						&= f(x_1\vec{e}_1)_+f(x_2\vec{e}_2)+f(x_3\vec{e}_3) \\
						&= x_1 f(\vec{e}_1)_+x_2 f(\vec{e}_2)+x_3 f(\vec{e}_3)
						\qquad\scriptstyle{sustituyo}\ f(\vec{e}_i)\\
						&= x_1 (\vec{u}_1+\vec{u}_2)+x_2(\vec{u}_1-\vec{u}_2)+x_3(2\vec{u}_1)
	\end{align*}

	Agrupo por los vectores de la base imagen $\vec{u}_1$ y $\vec{u}_2$
	$$(y_1,y_2)=\vec{u}_1(x_1+x_2+2x_3)+\vec{u}_2(x_1-x_2) \Rightarrow
	\begin{cases}
		y_1=x_1+x_2+2x_3 \\
		y_2=x_1-x_2
	\end{cases}$$

	Así que las ecuaciones implícitas del nucleo son
	$$\begin{cases}
		x_1+x_2+2x_3=0 \\
		x_1-x_2=0
	\end{cases}$$

	Y la matriz que buscamos viene dada por

	$$(x_1,x_2,x_3)
	\begin{pmatrix}
		1 & \phantom{-}1 \\
		1 & -1\\
		2 & \phantom{-}0
	\end{pmatrix}
	=(x_1+x_2+2x_3,\ x_1-x_2)=(y_1, y_2)$$
	
	La respuesta es
	
	$$T=\begin{pmatrix}
		1 & \phantom{-}1 \\
		1 & -1\\
		2 & \phantom{-}0
	\end{pmatrix}$$

	\paragraph{b. Ecuación y dimensiones de $N(f)$ y de $Im(f)$} El núcleo del homomorfismo se obtiene de
	$$(x_1,x_2,x_3)
	\begin{pmatrix}
		1 & \phantom{-}1 \\
		1 & -1\\
		2 & \phantom{-}0
	\end{pmatrix}=(0,0,0)$$
	Que da lugar a
	$$
	\begin{cases}
		x_1+x_2+2x_3=0 \\
		x_1-x_2=0
	\end{cases}
	\xrightarrow{\scriptstyle{x_2\ \text{como parámetro}}}
	\begin{cases}
		x_1+2x_3=-x_2=-\lambda \\
		x_1=x_2=\lambda
	\end{cases}$$
	Obtenemos $N(f)=\{(\lambda,\lambda,-\lambda)/\lambda\in\mathbb{R}\}$, y la base 
	de dicho núcleo esta formado por un único vector $B_N=\{(1,1,-1)\}$ quedando así claro que será 
	de dimensión 1. Vamos a comprobarlo
	\begin{align*}
		dim(\mathbb{R}^3) &= dim[N(f)]+dim[Im(f)] \quad \scriptstyle{\text{sustituyendo}}\\
		3 &= dim[N(f)] + 2
	\end{align*}
	Donde el primer miembro es la dimensión del conjunto origen y $dim[Im(f)]=Rango(T)=2$, con lo
	cual tenemos $dim[N(f)]=1$.

	\paragraph{Ecuaciones de $Im(f)$} Su dimensión y la del espacio vectorial que lo
	contiene, $\mathbb{R}^2$, son iguales, con lo cual, no está sujeto a ninguna ecuación implícita.

	\paragraph{c. Tipo de aplicación lineal} La aplicación es un homomorfismo de $V^3(\mathbb{R}^3)$ sobre
	$V^2(\mathbb{R})$, es decir, se trata de un epimorfismo.
\end{problema}

% ====================================================
\begin{problema}{\cite[12p349]{palacios}}{
	Mediante la matriz
	$T=\begin{pmatrix}
		1 & 0 & 0 \\
		0 & 1 & 1 \\
		1 & 1 & 0 \\
		1 & 1 & 1
	\end{pmatrix}$
	se establece un homomorfismo entre $V^4(\mathbb{R})$ y $W^3(\mathbb{R}^3)$. La matriz $T$ está
	referida a las bases $B_1=\{\vec{e}_1,\vec{e}_2,\vec{e}_3,\vec{e}_4\}$ y
	$B_2=\{\vec{u}_1,\vec{u}_2,\vec{u}_3\}$.

	Se pide

	\begin{enumerate}
		\item Ecuación matricial del homomorfismo
		\item Bases y dimensiones de $N(f)$ y de $Im(f)$
		\item Naturaleza del homomorfismo
	\end{enumerate}
	}
	\paragraph{a. Ecuación matricial} Es simplemente
	$$(x_1,x_2,x_3,x_4)
	\begin{pmatrix}
		1 & 0 & 0 \\
		0 & 1 & 1 \\
		1 & 1 & 0 \\
		1 & 1 & 1
	\end{pmatrix}
	=(y_1, y_2, y_3)
	\Rightarrow
	\begin{cases}
		y_1=x_1+x_3+x_4 \\
		y_2=x_2+x_3+x_4 \\
		y_3=x_2+x_4
	\end{cases}
	$$

	\paragraph{b. Bases y dimensiones del núcleo y la imagen} Para el núcleo igualamos $(y_1, y_2,
	y_3)=(0,0,0)$ en la expresión del apartado anterior. 

	$$(x_1,x_2,x_3,x_4)
	\begin{pmatrix}
		1 & 0 & 0 \\
		0 & 1 & 1 \\
		1 & 1 & 0 \\
		1 & 1 & 1
	\end{pmatrix}
	=(0,0,0)$$

	Desarrollamos el producto de matrices y como tenemos cuatro incógnitas y tres
	ecuaciones trataremos $x_4$ como si fuese un parámetro y obtenemos

	$$
	\begin{cases}
		x_1+x_3+x_4=0 \\
		x_2+x_3+x_4=0 \\
		x_2+x_4=0
	\end{cases}
	\xrightarrow{\scriptstyle{x_4=\lambda}}
	\begin{cases}
		x_1=x_2=x_4=-\lambda \\
		x_3=0
	\end{cases}
	$$

	Entonces el núcleo es $N(f)=\{(\lambda,\lambda,0,\lambda)/\lambda\in\mathbb{R}\}$ y su base
	tiene un único vector $B_N=\{(1,1,0,1)\}$. Su dimensión es 1.

	En cuanto a su imagen tendrá dimensión 3 puesto que $Rango(T)=dim[Im(f)]=3$. Queda confirmado por el hecho
	de que 
	\begin{align*}
		dim(\mathbb{R}^3) &= dim[N(f)]+dim[Im(f)] \quad \scriptstyle{\text{sustituyendo}}\\
		4 &= 1 + dim[Im(f)]
	\end{align*}

	Así pues, debemos buscar los tres vectores que formen la base de $Im(f)$ para poder formar todos
	los vectores $(y_1,y_2,y_3)$ mediante combinación lineal. Los hallaremos entre los vectores
	de la matriz de transformación $T$ que sean independientes, por ejemplo, los tres primeros.

	$$B_I=\{(1,0,0), (0,1,1), (1,1,0)\}$$

	\paragraph{c. Naturaleza del homomorfismo} Se trata de un epimorfismo, es decir, un homorfismo
	de $V^4(\mathbb{R})$ sobre $W^3(\mathbb{R})$, ya que la dimensión del subespacio $Im(f)$
	coincide con la del espacio de llegada $W^3(\mathbb{R})$.
\end{problema}


% ================================

\begin{problema}{\cite[89p139]{prieto2000}}{
	Sea $\vec{w}$ un vector de un espacio vectorial $E$ sobre un cuerpo $\mathbb{K}$. Demostrar
	la siguiente equivalencia:
	$$\mathbb{K}\vec{w}=\{0\} \iff \vec{w}=\vec{0}$$
	}
	Ambos conjunto son iguales. Sabemos que para el primer conjunto se cumple
	$1\cdot\vec{w}\in\mathbb{K}\vec{w}=\{\vec{0}\}$ luego $1\vec{w}=\vec{0}$ esto solo ocurrirá
	cuando $\vec{w}=\vec{0}$. 

	Por otra parte si $\vec{w}=\vec{0} \Rightarrow \mathbb{K}\vec{w}=\{\vec{0}\}$ sustituyendo
	tenemos
	$\mathbb{K}\cdot\vec{0}=\{\lambda\cdot\vec{0}/\lambda\in\mathbb{K}\}=\{\vec{0}\}\;\blacksquare$
\end{problema}


\printbibliography


\end{document}

